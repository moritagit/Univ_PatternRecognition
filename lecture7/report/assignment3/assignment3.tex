\documentclass[class=jsarticle, crop=false, dvipdfmx, fleqn]{standalone}
\input{/Users/User/Documents/University/report_template/preamble/preamble}
\begin{document}

\section*{宿題3}

MNISTを識別するニューラルネットワークを実装する。
宿題1と同等の構造を利用する。

中間層のユニット数を128,
学習率を0.001,ミニバッチサイズを32,エポックを50とした。
訓練データのうち10,000個をvalidation用に分け,
学習状況の確認に利用した。

混同行列は表\ref{tab:confusion_matrix_NN}のようになり,
また,各カテゴリごとの正解率等は表\ref{tab:result_NN}のようになった。
結果としては,学習は進まず,全て3と予測してしまうモデルとなってしまった。
ハイパラを振ったりもしてみたが,
1つの数字を予測するのみのモデルとなってしまった。
原因としては,まず実装ミスを疑ったが,
宿題1と同じモジュールを使っているのであまり考えにくいと思われた。
現状,ハイパラにかなり敏感なのかという考察となった。

プログラムは\pageref{listing:assignment23}ページのListing \ref{listing:assignment23}に示した。
関数についての説明は宿題2に示した通りである。
また,モデルについては宿題1で示した通りである。

\begin{table}[H]
	\centering
	\caption{NNに対する混同行列}
	\begin{tabular}{|c||cccccccccc|} \hline
			& 0 & 1 & 2 & 3 & 4 & 5 & 6 & 7 & 8 & 9 \\ \hline\hline
        0     & 0 & 0 & 0 & 980 & 0 & 0 & 0 & 0 & 0 & 0 \\
        1     & 0 & 0 & 0 & 1135 & 0 & 0 & 0 & 0 & 0 & 0 \\
        2     & 0 & 0 & 0 & 1032 & 0 & 0 & 0 & 0 & 0 & 0 \\
        3     & 0 & 0 & 0 & 1010 & 0 & 0 & 0 & 0 & 0 & 0 \\
        4     & 0 & 0 & 0 & 982 & 0 & 0 & 0 & 0 & 0 & 0 \\
        5     & 0 & 0 & 0 & 892 & 0 & 0 & 0 & 0 & 0 & 0 \\
        6     & 0 & 0 & 0 & 958 & 0 & 0 & 0 & 0 & 0 & 0 \\
        7     & 0 & 0 & 0 & 1028 & 0 & 0 & 0 & 0 & 0 & 0 \\
        8     & 0 & 0 & 0 & 974 & 0 & 0 & 0 & 0 & 0 & 0 \\
        9     & 0 & 0 & 0 & 1009 & 0 & 0 & 0 & 0 & 0 & 0 \\
		\hline
	\end{tabular}
	\label{tab:confusion_matrix_NN}
\end{table}

\begin{table}[H]
	\centering
	\caption{NNに対する各カテゴリごとの結果}
	\begin{tabular}{crrr}
		Category & {\#}Data & {\#}Correct & Accuracy \\ \hline
        0 & 980 & 0 & 0.000 \\
        1 & 1135 & 0 & 0.000 \\
        2 & 1032 & 0 & 0.000 \\
        3 & 1010 & 1010 & 1.000 \\
        4 & 982 & 0 & 0.000 \\
        5 & 892 & 0 & 0.000 \\
        6 & 958 & 0 & 0.000 \\
        7 & 1028 & 0 & 0.000 \\
        8 & 974 & 0 & 0.000 \\
        9 & 1009 & 0 & 0.000 \\
        All & 10000 & 1010 & 0.101 \\
	\end{tabular}
	\label{tab:result_NN}
\end{table}


\end{document}
