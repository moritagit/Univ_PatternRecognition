\documentclass[class=jsarticle, crop=false, dvipdfmx, fleqn]{standalone}
\input{/Users/User/Documents/University/report_template/preamble/preamble}
\begin{document}

\section*{宿題3}

MNISTを識別するニューラルネットワークを実装する。
宿題1と同等の構造を利用する。

中間層のユニット数を1,000,
学習率を0.007,ミニバッチサイズを100,エポックを500とした。
訓練データのうち10,000個をvalidation用に分け,
学習状況の確認に利用した。

混同行列は表\ref{tab:confusion_matrix_NN}のようになり,
また,各カテゴリごとの正解率等は表\ref{tab:result_NN}のようになった。

なお,プログラムは\pageref{listing:assignment23}ページのListing \ref{listing:assignment23}に示した。
関数についての説明は宿題2に示した通りである。
また,モデルについては宿題1で示した通りである。

\begin{table}[H]
	\centering
	\caption{NNに対する混同行列}
	\begin{tabular}{|c||cccccccccc|} \hline
			& 0 & 1 & 2 & 3 & 4 & 5 & 6 & 7 & 8 & 9 \\ \hline\hline
        0     & 957 & 0 & 0 & 2 & 0 & 6 & 7 & 2 & 6 & 0 \\
        1     & 0 & 1101 & 3 & 1 & 3 & 1 & 5 & 1 & 20 & 0 \\
        2     & 33 & 51 & 802 & 27 & 21 & 1 & 32 & 20 & 40 & 5 \\
        3     & 9 & 18 & 24 & 875 & 7 & 19 & 4 & 19 & 25 & 10 \\
        4     & 1 & 16 & 7 & 1 & 895 & 3 & 8 & 1 & 10 & 40 \\
        5     & 34 & 16 & 3 & 74 & 25 & 656 & 16 & 16 & 36 & 16 \\
        6     & 40 & 9 & 12 & 0 & 26 & 17 & 843 & 0 & 11 & 0 \\
        7     & 5 & 39 & 11 & 7 & 26 & 1 & 0 & 890 & 4 & 45 \\
        8     & 20 & 43 & 7 & 23 & 31 & 34 & 13 & 13 & 771 & 19 \\
        9     & 19 & 11 & 2 & 12 & 85 & 1 & 0 & 83 & 6 & 790 \\
		\hline
	\end{tabular}
	\label{tab:confusion_matrix_NN}
\end{table}

\begin{table}[H]
	\centering
	\caption{NNに対する各カテゴリごとの結果}
	\begin{tabular}{crrr}
		Category & {\#}Data & {\#}Correct & Accuracy \\ \hline
        0 & 980 & 957 & 0.977 \\
        1 & 1,135 & 1101 & 0.970 \\
        2 & 1,032 & 802 & 0.777 \\
        3 & 1,010 & 875 & 0.866 \\
        4 & 9,82 & 895 & 0.911 \\
        5 & 8,92 & 656 & 0.735 \\
        6 & 9,58 & 843 & 0.880 \\
        7 & 1,028 & 890 & 0.866 \\
        8 & 9,74 & 771 & 0.792 \\
        9 & 1,009 & 790 & 0.783 \\
        All & 10,000 & 8,580 & 0.858 \\
	\end{tabular}
	\label{tab:result_NN}
\end{table}


\end{document}
