\documentclass[class=jsarticle, crop=false, dvipdfmx, fleqn]{standalone}
\input{/Users/User/Documents/University/report_template/preamble/preamble}
\begin{document}
\section*{宿題2}


\begin{comment}
\subsection*{問題1}

\(x\)が正規分布に従うとき,
\(y = x^2\)を考えると,
\(x\)と\(y\)は無相関であることと,
独立ではないことを示す。




\subsection*{問題2}
\end{comment}


次式を導出する。
ここで,\(\bm{Z} \in \mathbb{R}^d,\ \bm{\Sigma} \in \mathbb{R}^{d \times d},\ \mu \in \mathbb{R}\)である。
\begin{equation}
    \pdv{}{\bm{Z}} \qty{\bm{Z}^\mathrm{T} \bm{\Sigma}^{-1} \bm{Z} - \mu(\bm{Z}^\mathrm{T}\bm{Z} - 1)} = 2 \bm{\Sigma}^{-1} \bm{Z} - 2\mu\bm{Z}
\end{equation}

\(\bm{Z},\ \bm{\Sigma}\)を次のようにおく。
\begin{align}
    & \bm{Z} = \begin{bmatrix} z_1 \\ \vdots \\ z_d \end{bmatrix} \\
    & \bm{\Sigma}^{-1} = \bm{\Lambda} =
        \begin{bmatrix}
            \lambda_{11} & \cdots & \lambda_{1d} \\
            \vdots & \ddots & \vdots \\
            \lambda_{d1} & \cdots & \lambda{dd}
        \end{bmatrix}
\end{align}
このとき,
\begin{align}
    \bm{Z}^\mathrm{T} \bm{Z}
        & =
            \begin{bmatrix} z_1 & \cdots & z_d \end{bmatrix}
            \begin{bmatrix} z_1 \\ \vdots \\ z_d \end{bmatrix} \notag \\
        & = {z_1}^2 + \cdots + {z_d}^2 \notag \\
        & = \sum_{i=1}^{d} {z_i}^2
\end{align}
\begin{align}
    \bm{\Sigma}^{-1} \bm{Z}
        & = \bm{\Lambda} \bm{Z} \notag \\
        & =
            \begin{bmatrix}
                \lambda_{11} & \cdots & \lambda_{1d} \\
                \vdots & \ddots & \vdots \\
                \lambda_{d1} & \cdots & \lambda_{dd}
            \end{bmatrix}
            \begin{bmatrix} z_1 \\ \vdots \\ z_d \end{bmatrix} \notag \\
        & =
            \begin{bmatrix}
                \lambda_{11} z_1 + \cdots + \lambda_{1d} z_d \\
                \vdots \\
                \lambda_{d1} z_1 + \cdots + \lambda_{dd} z_d \\
            \end{bmatrix} \notag \\
        & =
            \begin{bmatrix}
                \sum_{j=1}^{d} \lambda_{1j} z_j \\
                \vdots \\
                \sum_{j=1}^{d} \lambda_{dj} z_j \\
            \end{bmatrix}
\end{align}
\begin{align}
    \bm{Z}^\mathrm{T} \bm{\Sigma}^{-1} \bm{Z}
        & = \bm{Z}^\mathrm{T} \bm{\Lambda} \bm{Z} \notag \\
        & = \begin{bmatrix} z_1 & \cdots & z_d \end{bmatrix}
            \begin{bmatrix}
                \sum_{j=1}^{d} \lambda_{1j} z_j \\
                \vdots \\
                \sum_{j=1}^{d} \lambda_{dj} z_j \\
            \end{bmatrix} \notag \\
        & = z_1 \sum_{j=1}^{d} \lambda_{1j} z_j + \cdots + z_d \sum_{j=1}^{d} \lambda_{dj} z_j \notag \\
        & = \sum_{i=1}^{d} z_i \sum_{j=1}^{d} \lambda_{ij} z_j \\
        & = \sum_{i=1}^{d} \sum_{j=1}^{d} \lambda_{ij} z_i z_j \\
        & = \sum_{i=1}^{d} \lambda_{ii} {z_i}^2 + \sum_{i=1}^{d-1} \sum_{j=i+1}^{d} \lambda_{ij} z_i z_j + \sum_{j=1}^{d-1} \sum_{i=j+1}^{d} \lambda_{ij} z_i z_j \\
        & = \sum_{i=1}^{d} \lambda_{ii} {z_i}^2 + 2\sum_{i=1}^{d-1} \sum_{j=i+1}^{d} \lambda_{ij} z_i z_j
\end{align}
である。
ここで,\(\bm{\Sigma}\)は対称行列であるから,
\(\bm{\Lambda} = \bm{\Sigma}^{-1}\)も対称行列であり,
\(\lambda_{ij} = \lambda_{ji}\)となることを用いた。
このとき,
\begin{equation}
    \pdv{}{z_k} \qty(\bm{Z}^\mathrm{T} \bm{Z})
        = \pdv{}{z_k} \qty(\sum_{i=1}^{d} {z_k}^2)
        = 2 z_k
\end{equation}
\begin{align}
    \pdv{}{z_k} \qty(\bm{Z}^\mathrm{T} \bm{\Sigma}^{-1} \bm{Z})
        & = \pdv{}{z_k} \qty(\sum_{i=1}^{d} \sum_{j=1}^{d} \lambda_{ij} z_i z_j) \\
        & = \pdv{}{z_k} \qty(\sum_{i=1}^{d} \lambda_{ii} {z_i}^2 + 2\sum_{i=1}^{d-1} \sum_{j=i+1}^{d} \lambda_{ij} z_i z_j) \\
        & = 2 \lambda_{kk} z_k + 2\qty(\sum_{j=k+1}^{d} \lambda_{kj} z_j + \sum_{i=1}^{k-1} \lambda_{ik} z_i) \\
        & = 2\sum_{l=1}^{d} \lambda_{kl} z_l
\end{align}
であるから,
\begin{align}
    \pdv{}{\bm{Z}} \qty(\bm{Z}^\mathrm{T} \bm{Z})
        =
            \begin{bmatrix}
                \pdv{}{z_1} \qty(\bm{Z}^\mathrm{T} \bm{Z}) \\
                \vdots \\
                \pdv{}{z_d} \qty(\bm{Z}^\mathrm{T} \bm{Z}) \\
            \end{bmatrix}
        =
            \begin{bmatrix}
                2z_1 \\ \vdots \\ 2z_d
            \end{bmatrix}
        = 2 \bm{Z}
\end{align}
\begin{align}
    \pdv{}{\bm{Z}} \qty(\bm{Z}^\mathrm{T} \bm{\Sigma}^{-1} \bm{Z})
        & =
            \begin{bmatrix}
                \pdv{}{z_1} \qty(\bm{Z}^\mathrm{T} \bm{\Sigma}^{-1} \bm{Z}) \\
                \vdots \\
                \pdv{}{z_d} \qty(\bm{Z}^\mathrm{T} \bm{\Sigma}^{-1} \bm{Z}) \\
            \end{bmatrix} \\
        & =
            \begin{bmatrix}
                2\sum_{l=1}^{d} \lambda_{1l} z_l \\
                \vdots \\
                2\sum_{l=1}^{d} \lambda_{dl} z_l
            \end{bmatrix} \\
        & = 2 \bm{\Sigma}^{-1} \bm{Z}
\end{align}
となる。
また,当然\(\pdv*{\mu}{\bm{Z}} = \bm{0}\)である。
以上より,
\begin{align}
    \pdv{}{\bm{Z}} \qty{\bm{Z}^\mathrm{T} \bm{\Sigma}^{-1} \bm{Z} - \mu(\bm{Z}^\mathrm{T}\bm{Z} - 1)}
        & = \pdv{}{\bm{Z}} \qty(\bm{Z}^\mathrm{T} \bm{\Sigma}^{-1} \bm{Z}) - \mu \pdv{}{\bm{Z}} \qty(\bm{Z}^\mathrm{T} \bm{Z}) \\
        & = 2 \bm{\Sigma}^{-1} \bm{Z} - 2\mu\bm{Z}
\end{align}



\end{document}
