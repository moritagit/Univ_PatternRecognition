\documentclass[class=jsarticle, crop=false, dvipdfmx, fleqn]{standalone}
\input{/Users/User/Documents/University/report_template/preamble/preamble}
\begin{document}
\section*{宿題2}


\subsection*{(1)}

二つの超平面
\begin{align}
	& \text{平面1:} \quad \bm{w}^\mathrm{T} \bm{x} - b = 1 \\
	& \text{平面2:} \quad \bm{w}^\mathrm{T} \bm{x} - b = -1
\end{align}
の間のマージンが
\begin{equation}
	\frac{2}{||\bm{w}||}
\end{equation}
であることを示す。

それぞれ平面1,平面2上にあり,
かつその差分ベクトルがその二平面に垂直であるような
2点\(\bm{x}_1\),\(\bm{x}_2\)を考える。
つまり,
\begin{align}
	& \bm{w}^\mathrm{T} \bm{x}_1 - b = 1 \\
	& \bm{w}^\mathrm{T} \bm{x}_2 - b = -1 \\
	& \bm{m} = \bm{x}_1 - \bm{x}_2 = k \bm{w}
		\qquad (k \in \mathbb{R}_{\neq 0})
\end{align}
これらより,
\begin{align}
	\bm{w}^\mathrm{T} (\bm{x}_1 - \bm{x}_2)
		& = \bm{w}^\mathrm{T} \bm{m} \\
		& = \bm{w}^\mathrm{T} \cdot k \bm{w} \\
		& = k ||\bm{w}||^2 \\
		& = 2
\end{align}
となる。
よって,
\begin{align}
	& k = \frac{2}{||\bm{w}||^2} \\
	& \bm{m} = k \bm{w} = \frac{2}{||\bm{w}||^2} \bm{w}
\end{align}
となるから,結局,二平面の間のマージン\(||\bm{m}||\)は,
\begin{equation}
	||\bm{m}|| = \frac{2}{||\bm{w}||}
\end{equation}
となる。


\clearpage
\subsection*{(2)}

ソフトマージンSVMの主形式
\begin{align}
	& \arg\min_{\bm{w},\ b,\ \bm{\xi}} \qty{\frac{1}{2} ||\bm{w}||^2 + C \sum_{i}^{n} \xi_i} \\
\intertext{subject to}
	& y_i (\bm{w}^\mathrm{T} \bm{x}_i - b) \ge 1 - \xi_i \\
	& \xi \ge 0
\end{align}
から,双対形式を導く。

一般に,主問題
\begin{align}
	& f^{*} = \min_{x \in \Xi} f(\bm{x}) \\
\intertext{subject to}
	& \bm{g}(\bm{x}) = \bm{0} \\
	& \bm{h}(\bm{x}) \le \bm{0}
\end{align}
に対するLagrange関数は,
\begin{equation}
	L(\bm{x},\ \bm{\lambda},\ \bm{\mu}) = f(\bm{x}) + \bm{\lambda}^\mathrm{T} \bm{g}(\bm{x}) + \bm{\mu}^\mathrm{T} \bm{h}(\bm{x})
\end{equation}
となり,
Lagrange双対問題は,
\begin{align}
	& f^{*} = \max_{\bm{\lambda,\ \bm{\mu}}} \mathrm{inf}_{\bm{x} \in \bm{X}} L(\bm{x},\ \bm{\lambda},\ \bm{\mu}) \\
\intertext{subject to}
	& \bm{\mu} \ge \bm{0}
\end{align}
となる。
また,相補性条件より,
\begin{equation}
	\bm{\mu} \odot \bm{h} = \bm{0}
\end{equation}
が成立する。
ここで,\(\odot\)は要素ごとの積を表す。

これを今回の条件に適用すると,
Lagrange関数は,
\begin{align}
	& L(\bm{w},\ b,\ \bm{\alpha},\ \bm{\beta})
		= \frac{1}{2} ||\bm{w}||^2 + C \sum_{i}^{n} \xi_i
			- \sum_{i}^{n} \alpha_i \qty{y_i (\bm{w}^\mathrm{T} \bm{x}_i - b) - 1 + \xi_i}
			- \sum_{i}^{n} \beta_i \xi_i
	\label{eq:lagrangian} \\
\intertext{subject to}
	& \bm{\alpha} \ge \bm{0} \label{eq:cond_alpha} \\
	& \bm{\beta} \ge \bm{0} \label{eq:cond_beta} \\
	& \alpha_i \qty{y_i (\bm{w}^\mathrm{T} \bm{x}_i - b) - 1 + \xi_i} = 0 \label{eq:cond_comple_alpha} \\
	& \beta_i \xi_i = 0 \label{eq:cond_comple_beta}
\end{align}
\(L\)の\(\bm{w},\ b,\ \bm{\xi}\)による偏微分を考える。
\begin{align}
	\pdv{L}{\bm{w}}
		& = \bm{w} - \sum_{i}^{n} \alpha_i y_i \bm{x}_i
		= \bm{0} \\
	\hat{\bm{w}} & = \sum_{i}^{n} \alpha_i y_i \bm{x}_i
	\label{eq:w_hat}
\end{align}
また,
\begin{equation}
	\pdv{L}{b}
		= \sum_{i}^{n} \alpha_i y_i
		= 0
	\label{eq:pdv_b}
\end{equation}
また,
\begin{equation}
	\pdv{L}{\xi_i}
		= C - \alpha_i - \beta_i
		= 0
\end{equation}
から,
\begin{equation}
	\alpha_i + \beta_i = C
	\label{eq:alpha_beta_c}
\end{equation}

式(\ref{eq:lagrangian})のLagrange関数を,
式(\ref{eq:w_hat}),(\ref{eq:pdv_b}),(\ref{eq:alpha_beta_c})を用いて変形する。
\begin{align}
	L(\bm{w},\ b,\ \bm{\alpha},\ \bm{\beta})
		& = \frac{1}{2} ||\bm{w}||^2 + C \sum_{i}^{n} \xi_i
			- \sum_{i}^{n} \alpha_i \qty{y_i (\bm{w}^\mathrm{T} \bm{x}_i - b) - 1 + \xi_i}
			- \sum_{i}^{n} \beta_i \xi_i\\
		& = \frac{1}{2} ||\bm{w}||^2 - \bm{w}^\mathrm{T} \sum_{i}^{n} \alpha_i y_i \bm{x}_i + + b \sum_{i}^{n} \alpha_i y_i \sum_{i}^{n} \alpha_i + \sum_{i}^{n} (C - \alpha_i - \beta_i) \xi_i \\
		& = \frac{1}{2} ||\sum_{i}^{n} \alpha_i y_i \bm{x}_i||^2 - \sum_{j}^{n} \alpha_j y_j \bm{x}_j^\mathrm{T} \sum_{i}^{n} \alpha_i y_i \bm{x}_i + \sum_{i}^{n} \alpha_i \\
		& = \sum_{i}^{n} \alpha_i - \frac{1}{2} \sum_{i, j}^{n} \alpha_i \alpha_j y_i y_j \bm{x}_i^\mathrm{T} \bm{x}_j
\end{align}
よって,\(\bm{w},\ \bm{\xi},\ \bm{\beta}\)がLagrange関数から消去されたので,結局,
\begin{align}
	& L(\bm{\alpha}) = \sum_{i}^{n} \alpha_i - \frac{1}{2} \sum_{i, j}^{n} \alpha_i \alpha_j y_i y_j \bm{x}_i^\mathrm{T} \bm{x}_j \\
	& \bm{\alpha} = \arg\max_{\bm{\alpha}} L(\bm{\alpha})
\end{align}
が解くべき問題となる。
また,式(\ref{eq:cond_alpha}),(\ref{eq:cond_beta}),(\ref{eq:alpha_beta_c})から,
\begin{align}
	0 \le \alpha_i \le C \\
	0 \le \beta_i \le C
\end{align}
なので,式(\ref{eq:pdv_b})と合わせて,制約は,
\begin{align}
	0 \le \alpha_i \le C \\
	\sum_{i}^{n} \alpha_i y_i = 0
\end{align}
となる。


\end{document}
