\documentclass[class=jsarticle, crop=false, dvipdfmx, fleqn]{standalone}
\input{/Users/User/Documents/University/report_template/preamble/preamble}
\begin{document}
\section{プログラム}

実行環境と用いた言語・ライブラリを以下の表\ref{tab:cp_env}に示す。

\begin{table}[H]
    \centering
    \caption{プログラムの実行環境}
    \begin{tabular}{lcl}
        OS & : & Microsoft Windows 10 Pro (64bit) \\
        CPU & : & Intel(R) Core(TM) i5-4300U \\
        RAM & : & 4.00 GB \\
        使用言語 & : & Python3.6 \\
        可視化 & : & matplotlibライブラリ
    \end{tabular}
    \label{tab:cp_env}
\end{table}


ソースコードはListing \ref{listing:assignment1}に示した。
以下に簡単に各関数の説明を記す。

\begin{itemize}
    \item load\_data \\
        .matファイルからデータを取り出す。
    \item plot \\
        点群及び境界をプロットする。
        2クラスをoとxで表し,
        分類結果が正しいものを青,
        誤っているものを赤で示す。
    \item perceptron \\
        パーセプトロンを用いて重みを求める関数。
    \item mse \\
        MSE法を用いて重みを求める関数。
        LMS法を用いる場合と解析的に求める場合とを使い分けられる。
\end{itemize}

\reportlisting[listing:assignment1]{assignment1.py}{../program/assignment1.py}

\end{document}
